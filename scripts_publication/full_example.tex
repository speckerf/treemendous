% Options for packages loaded elsewhere
\PassOptionsToPackage{unicode}{hyperref}
\PassOptionsToPackage{hyphens}{url}
%
\documentclass[
  11pt,
]{article}
\usepackage{amsmath,amssymb}
\usepackage{iftex}
\ifPDFTeX
  \usepackage[T1]{fontenc}
  \usepackage[utf8]{inputenc}
  \usepackage{textcomp} % provide euro and other symbols
\else % if luatex or xetex
  \usepackage{unicode-math} % this also loads fontspec
  \defaultfontfeatures{Scale=MatchLowercase}
  \defaultfontfeatures[\rmfamily]{Ligatures=TeX,Scale=1}
\fi
\usepackage{lmodern}
\ifPDFTeX\else
  % xetex/luatex font selection
\fi
% Use upquote if available, for straight quotes in verbatim environments
\IfFileExists{upquote.sty}{\usepackage{upquote}}{}
\IfFileExists{microtype.sty}{% use microtype if available
  \usepackage[]{microtype}
  \UseMicrotypeSet[protrusion]{basicmath} % disable protrusion for tt fonts
}{}
\makeatletter
\@ifundefined{KOMAClassName}{% if non-KOMA class
  \IfFileExists{parskip.sty}{%
    \usepackage{parskip}
  }{% else
    \setlength{\parindent}{0pt}
    \setlength{\parskip}{6pt plus 2pt minus 1pt}}
}{% if KOMA class
  \KOMAoptions{parskip=half}}
\makeatother
\usepackage{xcolor}
\usepackage[margin=1in]{geometry}
\usepackage{color}
\usepackage{fancyvrb}
\newcommand{\VerbBar}{|}
\newcommand{\VERB}{\Verb[commandchars=\\\{\}]}
\DefineVerbatimEnvironment{Highlighting}{Verbatim}{commandchars=\\\{\}}
% Add ',fontsize=\small' for more characters per line
\usepackage{framed}
\definecolor{shadecolor}{RGB}{248,248,248}
\newenvironment{Shaded}{\begin{snugshade}}{\end{snugshade}}
\newcommand{\AlertTok}[1]{\textcolor[rgb]{0.94,0.16,0.16}{#1}}
\newcommand{\AnnotationTok}[1]{\textcolor[rgb]{0.56,0.35,0.01}{\textbf{\textit{#1}}}}
\newcommand{\AttributeTok}[1]{\textcolor[rgb]{0.13,0.29,0.53}{#1}}
\newcommand{\BaseNTok}[1]{\textcolor[rgb]{0.00,0.00,0.81}{#1}}
\newcommand{\BuiltInTok}[1]{#1}
\newcommand{\CharTok}[1]{\textcolor[rgb]{0.31,0.60,0.02}{#1}}
\newcommand{\CommentTok}[1]{\textcolor[rgb]{0.56,0.35,0.01}{\textit{#1}}}
\newcommand{\CommentVarTok}[1]{\textcolor[rgb]{0.56,0.35,0.01}{\textbf{\textit{#1}}}}
\newcommand{\ConstantTok}[1]{\textcolor[rgb]{0.56,0.35,0.01}{#1}}
\newcommand{\ControlFlowTok}[1]{\textcolor[rgb]{0.13,0.29,0.53}{\textbf{#1}}}
\newcommand{\DataTypeTok}[1]{\textcolor[rgb]{0.13,0.29,0.53}{#1}}
\newcommand{\DecValTok}[1]{\textcolor[rgb]{0.00,0.00,0.81}{#1}}
\newcommand{\DocumentationTok}[1]{\textcolor[rgb]{0.56,0.35,0.01}{\textbf{\textit{#1}}}}
\newcommand{\ErrorTok}[1]{\textcolor[rgb]{0.64,0.00,0.00}{\textbf{#1}}}
\newcommand{\ExtensionTok}[1]{#1}
\newcommand{\FloatTok}[1]{\textcolor[rgb]{0.00,0.00,0.81}{#1}}
\newcommand{\FunctionTok}[1]{\textcolor[rgb]{0.13,0.29,0.53}{\textbf{#1}}}
\newcommand{\ImportTok}[1]{#1}
\newcommand{\InformationTok}[1]{\textcolor[rgb]{0.56,0.35,0.01}{\textbf{\textit{#1}}}}
\newcommand{\KeywordTok}[1]{\textcolor[rgb]{0.13,0.29,0.53}{\textbf{#1}}}
\newcommand{\NormalTok}[1]{#1}
\newcommand{\OperatorTok}[1]{\textcolor[rgb]{0.81,0.36,0.00}{\textbf{#1}}}
\newcommand{\OtherTok}[1]{\textcolor[rgb]{0.56,0.35,0.01}{#1}}
\newcommand{\PreprocessorTok}[1]{\textcolor[rgb]{0.56,0.35,0.01}{\textit{#1}}}
\newcommand{\RegionMarkerTok}[1]{#1}
\newcommand{\SpecialCharTok}[1]{\textcolor[rgb]{0.81,0.36,0.00}{\textbf{#1}}}
\newcommand{\SpecialStringTok}[1]{\textcolor[rgb]{0.31,0.60,0.02}{#1}}
\newcommand{\StringTok}[1]{\textcolor[rgb]{0.31,0.60,0.02}{#1}}
\newcommand{\VariableTok}[1]{\textcolor[rgb]{0.00,0.00,0.00}{#1}}
\newcommand{\VerbatimStringTok}[1]{\textcolor[rgb]{0.31,0.60,0.02}{#1}}
\newcommand{\WarningTok}[1]{\textcolor[rgb]{0.56,0.35,0.01}{\textbf{\textit{#1}}}}
\usepackage{graphicx}
\makeatletter
\def\maxwidth{\ifdim\Gin@nat@width>\linewidth\linewidth\else\Gin@nat@width\fi}
\def\maxheight{\ifdim\Gin@nat@height>\textheight\textheight\else\Gin@nat@height\fi}
\makeatother
% Scale images if necessary, so that they will not overflow the page
% margins by default, and it is still possible to overwrite the defaults
% using explicit options in \includegraphics[width, height, ...]{}
\setkeys{Gin}{width=\maxwidth,height=\maxheight,keepaspectratio}
% Set default figure placement to htbp
\makeatletter
\def\fps@figure{htbp}
\makeatother
\setlength{\emergencystretch}{3em} % prevent overfull lines
\providecommand{\tightlist}{%
  \setlength{\itemsep}{0pt}\setlength{\parskip}{0pt}}
\setcounter{secnumdepth}{-\maxdimen} % remove section numbering
\ifLuaTeX
  \usepackage{selnolig}  % disable illegal ligatures
\fi
\IfFileExists{bookmark.sty}{\usepackage{bookmark}}{\usepackage{hyperref}}
\IfFileExists{xurl.sty}{\usepackage{xurl}}{} % add URL line breaks if available
\urlstyle{same}
\hypersetup{
  pdftitle={Treemendous: Example},
  pdfauthor={Andrea Paz, Felix Specker, Dan Maynard},
  hidelinks,
  pdfcreator={LaTeX via pandoc}}

\title{Treemendous: Example}
\author{Andrea Paz, Felix Specker, Dan Maynard}
\date{2023-07-17}

\begin{document}
\maketitle

\hypertarget{package-installation}{%
\subsection{Package Installation}\label{package-installation}}

\begin{Shaded}
\begin{Highlighting}[]
\FunctionTok{library}\NormalTok{(devtools)}
\FunctionTok{install\_github}\NormalTok{(}\StringTok{"speckerf/treemendous"}\NormalTok{)}
\end{Highlighting}
\end{Shaded}

\hypertarget{species-list-preparation}{%
\subsection{Species List Preparation}\label{species-list-preparation}}

All functions of \textit{Treemendous} require the species name to be
split into two columns, \verb|Genus| and \verb|Species|, with the former
being capitalized. Assume you have two species,
\textit{Acer platanoides} and \textit{Fagus sylvatica}, you can create
the input \verb|tibble| by calling:

\begin{Shaded}
\begin{Highlighting}[]
\DocumentationTok{\#\#\# Species list preparation}
\FunctionTok{library}\NormalTok{(tidyverse)}
\NormalTok{species }\OtherTok{\textless{}{-}} \FunctionTok{c}\NormalTok{(}\StringTok{\textquotesingle{}Acer platanoides\textquotesingle{}}\NormalTok{, }\StringTok{\textquotesingle{}Fagus sylvatica\textquotesingle{}}\NormalTok{)}
\NormalTok{input }\OtherTok{\textless{}{-}}\NormalTok{ species }\SpecialCharTok{\%\textgreater{}\%}
\NormalTok{  tibble}\SpecialCharTok{::}\FunctionTok{as\_tibble\_col}\NormalTok{(}\AttributeTok{column\_name =} \StringTok{\textquotesingle{}binomial\textquotesingle{}}\NormalTok{) }\SpecialCharTok{\%\textgreater{}\%}
\NormalTok{  tidyr}\SpecialCharTok{::}\FunctionTok{separate}\NormalTok{(}\AttributeTok{col =} \StringTok{\textquotesingle{}binomial\textquotesingle{}}\NormalTok{, }\AttributeTok{into =} \FunctionTok{c}\NormalTok{(}\StringTok{\textquotesingle{}Genus\textquotesingle{}}\NormalTok{, }\StringTok{\textquotesingle{}Species\textquotesingle{}}\NormalTok{))}
\NormalTok{input}
\end{Highlighting}
\end{Shaded}

\begin{verbatim}
## # A tibble: 2 x 2
##   Genus Species    
##   <chr> <chr>      
## 1 Acer  platanoides
## 2 Fagus sylvatica
\end{verbatim}

Other useful functions for creating the input \verb|tibble| include:

\begin{Shaded}
\begin{Highlighting}[]
\NormalTok{readr}\SpecialCharTok{::}\FunctionTok{read\_csv}\NormalTok{(}\StringTok{\textquotesingle{}path\textquotesingle{}}\NormalTok{) }\CommentTok{\# import data}
\NormalTok{dplyr}\SpecialCharTok{::}\FunctionTok{select}\NormalTok{(Genus, Species) }\CommentTok{\# select columns}
\NormalTok{dplyr}\SpecialCharTok{::}\FunctionTok{distinct}\NormalTok{(Genus, Species) }\CommentTok{\# remove duplicate binomials}
\NormalTok{dplyr}\SpecialCharTok{::}\FunctionTok{rename}\NormalTok{(}\StringTok{\textquotesingle{}Genus\textquotesingle{}} \OtherTok{=} \StringTok{\textquotesingle{}old\_genus\_name\textquotesingle{}}\NormalTok{,}
                \StringTok{\textquotesingle{}Species\textquotesingle{}} \OtherTok{=} \StringTok{\textquotesingle{}old\_species\_name\textquotesingle{}}\NormalTok{) }\CommentTok{\# rename columns}
\NormalTok{dplyr}\SpecialCharTok{::}\FunctionTok{mutate}\NormalTok{(}\AttributeTok{Genus =}\NormalTok{ stringr}\SpecialCharTok{::}\FunctionTok{str\_to\_title}\NormalTok{(Genus)) }\CommentTok{\# capitalize Genus}
\NormalTok{dplyr}\SpecialCharTok{::}\FunctionTok{mutate}\NormalTok{(}\AttributeTok{Species =}\NormalTok{ stringr}\SpecialCharTok{::}\FunctionTok{str\_remove}\NormalTok{(Species, }\StringTok{".*?}\SpecialCharTok{\textbackslash{}\textbackslash{}}\StringTok{s"}\NormalTok{)) }\CommentTok{\# remove everything before first space}
\NormalTok{tidyr}\SpecialCharTok{::}\FunctionTok{drop\_na}\NormalTok{(}\FunctionTok{c}\NormalTok{(}\StringTok{\textquotesingle{}Genus\textquotesingle{}}\NormalTok{, }\StringTok{\textquotesingle{}Species\textquotesingle{}}\NormalTok{)) }\CommentTok{\# remove rows with NA\textquotesingle{}s}
\NormalTok{dplyr}\SpecialCharTok{::}\FunctionTok{arrange}\NormalTok{(Genus, Species) }\CommentTok{\# sort names}
\NormalTok{dplyr}\SpecialCharTok{::}\FunctionTok{bind\_rows}\NormalTok{(x, y) }\CommentTok{\# concatenate two tibble\textquotesingle{}s}
\end{Highlighting}
\end{Shaded}

\hypertarget{fia-standardize-species-names-from-the-u.s.-forest-inventory-and-analysis-program.}{%
\subsection{FIA: Standardize species names from the U.S. Forest
Inventory and Analysis
program.}\label{fia-standardize-species-names-from-the-u.s.-forest-inventory-and-analysis-program.}}

Along with the package comes an example dataset \verb|fia| with \(2171\)
different tree species names. Assume that we want to standardize these
species names according to a certain backbone (use the backbone
argument). The function \verb|summarize_output()| can be used to get a
summary of the process.

\begin{Shaded}
\begin{Highlighting}[]
\FunctionTok{library}\NormalTok{(treemendous)}
\end{Highlighting}
\end{Shaded}

\begin{Shaded}
\begin{Highlighting}[]
\NormalTok{result }\OtherTok{\textless{}{-}}\NormalTok{ fia }\SpecialCharTok{\%\textgreater{}\%} \FunctionTok{matching}\NormalTok{(}\AttributeTok{backbone =} \StringTok{\textquotesingle{}BGCI\textquotesingle{}}\NormalTok{)}
\FunctionTok{summarize\_output}\NormalTok{(result)}
\end{Highlighting}
\end{Shaded}

\begin{verbatim}
## [1] "matched: 1822 / 2171 were matched with 1822 distinct matched names."
## [2] "direct_match: 1779 / 2171"                                          
## [3] "indirectly matched: 43 / 392"                                       
## [4] "    genus_match: 313 / 392"                                         
## [5] "    fuzzy_match_genus: 2 / 79"                                      
## [6] "    direct_match_species_within_genus: 1 / 315"                     
## [7] "    suffix_match_species_within_genus: 11 / 314"                    
## [8] "    fuzzy_match_species_within_genus: 31 / 303"
\end{verbatim}

\noindent From \(2171\) species names in total, we were able to match
\(1822\) according to the backbone \verb|BGCI|, with \(1779\) names
matching exactly, and \(43\) species names matching using fuzzy- and
suffix-matching. Besides information about the matching process, the
output contains the old names (prefix \verb|Orig.|) as well as the
matched names (prefix \verb|Matched.|) as follows:

\begin{Shaded}
\begin{Highlighting}[]
\NormalTok{result }\SpecialCharTok{\%\textgreater{}\%} 
\NormalTok{  dplyr}\SpecialCharTok{::}\FunctionTok{slice\_head}\NormalTok{(}\AttributeTok{n=}\DecValTok{3}\NormalTok{) }\SpecialCharTok{\%\textgreater{}\%}
\NormalTok{  dplyr}\SpecialCharTok{::}\FunctionTok{select}\NormalTok{(}\DecValTok{1}\SpecialCharTok{:}\DecValTok{5}\NormalTok{)}
\end{Highlighting}
\end{Shaded}

\begin{verbatim}
## # A tibble: 3 x 5
##   Orig.Genus Orig.Species Matched.Genus Matched.Species matched
##   <chr>      <chr>        <chr>         <chr>           <lgl>  
## 1 Abies      amabilis     Abies         amabilis        TRUE   
## 2 Abies      balsamea     Abies         balsamea        TRUE   
## 3 Abies      bracteata    Abies         bracteata       TRUE
\end{verbatim}

\noindent We can further increase the number of matched species by using
the functions \verb|matching()| followed by \verb|enforce_matching()|.
Here, we specify the backbone \verb|BGCI|.

\begin{Shaded}
\begin{Highlighting}[]
\NormalTok{result }\OtherTok{\textless{}{-}}\NormalTok{ fia }\SpecialCharTok{\%\textgreater{}\%} 
  \FunctionTok{matching}\NormalTok{(}\AttributeTok{backbone =} \StringTok{\textquotesingle{}BGCI\textquotesingle{}}\NormalTok{) }\SpecialCharTok{\%\textgreater{}\%} 
  \FunctionTok{enforce\_matching}\NormalTok{(}\AttributeTok{backbone =} \StringTok{\textquotesingle{}BGCI\textquotesingle{}}\NormalTok{)}
\NormalTok{result }\SpecialCharTok{\%\textgreater{}\%} \FunctionTok{summarize\_output}\NormalTok{()}
\end{Highlighting}
\end{Shaded}

\begin{verbatim}
## [1] "matched: 2097 / 2171 were matched with 2036 distinct matched names."
## [2] "direct_match: 1779 / 2171"                                          
## [3] "indirectly matched: 43 / 392"                                       
## [4] "    genus_match: 93 / 117"                                          
## [5] "    fuzzy_match_genus: 2 / 24"                                      
## [6] "    direct_match_species_within_genus: 1 / 95"                      
## [7] "    suffix_match_species_within_genus: 11 / 94"                     
## [8] "    fuzzy_match_species_within_genus: 31 / 83"                      
## [9] "number of species matched via enforce_matching(): 275 / 349"
\end{verbatim}

\noindent Now, we are able to match \(2097\) species names in total,
with \(275\) species being matched via \verb|enforce_matching()|. Note
that the number of matched distinct species names is lower with
\(2044\), because several input species were matched to the same species
in the target database \verb|BGCI|.

\noindent Note that if we choose a different backbone than \verb|BGCI|,
then species can matched names that are not accepted (synonyms), we can
further resolve synonyms after matching the species names with the
function \verb|resolve_synonyms()|. Now, the output contains
additionally the accepted species names (prefix \verb|Accepted.|), as
well as a column \verb|Accepted.Backbone|, which states according to
which backbone the synonym was resolved.

\begin{Shaded}
\begin{Highlighting}[]
\NormalTok{result }\OtherTok{\textless{}{-}}\NormalTok{ fia }\SpecialCharTok{\%\textgreater{}\%} 
  \FunctionTok{matching}\NormalTok{(}\StringTok{\textquotesingle{}WFO\textquotesingle{}}\NormalTok{) }\SpecialCharTok{\%\textgreater{}\%} 
  \FunctionTok{resolve\_synonyms}\NormalTok{(}\StringTok{\textquotesingle{}WFO\textquotesingle{}}\NormalTok{)}
\end{Highlighting}
\end{Shaded}

\begin{Shaded}
\begin{Highlighting}[]
\NormalTok{result }\SpecialCharTok{\%\textgreater{}\%} 
\NormalTok{  dplyr}\SpecialCharTok{::}\FunctionTok{slice\_head}\NormalTok{(}\AttributeTok{n=}\DecValTok{3}\NormalTok{) }\SpecialCharTok{\%\textgreater{}\%} 
\NormalTok{  dplyr}\SpecialCharTok{::}\FunctionTok{select}\NormalTok{(dplyr}\SpecialCharTok{::}\FunctionTok{matches}\NormalTok{(}\StringTok{\textquotesingle{}Orig|Matched|Accepted\textquotesingle{}}\NormalTok{), }\SpecialCharTok{{-}}\StringTok{\textquotesingle{}matched\textquotesingle{}}\NormalTok{)}
\end{Highlighting}
\end{Shaded}

\begin{verbatim}
## # A tibble: 3 x 7
##   Orig.Genus Orig.Species Matched.Genus Matched.Species Accepted.Genus
##   <chr>      <chr>        <chr>         <chr>           <chr>         
## 1 Abies      amabilis     Abies         amabilis        Abies         
## 2 Abies      balsamea     Abies         balsamea        Abies         
## 3 Abies      bracteata    Abies         bracteata       Abies         
## # i 2 more variables: Accepted.Species <chr>, Accepted.Backbone <chr>
\end{verbatim}

\noindent Note that a warning message is produced ``Please consider
calling highlight\_flags() to investigate potential ambiguities upon
resolving synonyms to accepted names''. Potential ambiguities could have
been resolved in your dataset and it is suggested to use
\verb|highlight_flags()| to know more and decide if you want to check
them manually. The \verb|highlight_flags()| function should be used
separately from the others as it will only return species that have some
flag and not the full dataset. Note also that each entry can have
multiple flags:

\begin{Shaded}
\begin{Highlighting}[]
\NormalTok{flags }\OtherTok{\textless{}{-}}\NormalTok{ result }\SpecialCharTok{\%\textgreater{}\%} \FunctionTok{highlight\_flags}\NormalTok{(}\StringTok{\textquotesingle{}WFO\textquotesingle{}}\NormalTok{)}
\end{Highlighting}
\end{Shaded}

\begin{verbatim}
## In summary, 574 out of 2171 matched species have raised a flag.
\end{verbatim}

\begin{Shaded}
\begin{Highlighting}[]
\NormalTok{flags }\SpecialCharTok{\%\textgreater{}\%} 
\NormalTok{  dplyr}\SpecialCharTok{::}\FunctionTok{slice\_head}\NormalTok{(}\AttributeTok{n=}\DecValTok{3}\NormalTok{) }\SpecialCharTok{\%\textgreater{}\%} 
\NormalTok{  dplyr}\SpecialCharTok{::}\FunctionTok{select}\NormalTok{(dplyr}\SpecialCharTok{::}\FunctionTok{matches}\NormalTok{(}\StringTok{\textquotesingle{}Acc|ambiguity|link\textquotesingle{}}\NormalTok{))}
\end{Highlighting}
\end{Shaded}

\begin{verbatim}
## # A tibble: 3 x 6
##   Accepted.Genus Accepted.Species Accepted.Backbone WFO_authorship_ambiguity
##   <chr>          <chr>            <chr>             <lgl>                   
## 1 Abies          amabilis         WFO               TRUE                    
## 2 Abies          balsamea         WFO               FALSE                   
## 3 Abies          concolor         WFO               FALSE                   
## # i 2 more variables: WFO_infraspecific_ambiguity <lgl>,
## #   WFO_infraspecific_link <lgl>
\end{verbatim}

We can see the full breakdown of these flags as follows:

\begin{Shaded}
\begin{Highlighting}[]
\NormalTok{flags }\SpecialCharTok{\%\textgreater{}\%}\NormalTok{ dplyr}\SpecialCharTok{::}\FunctionTok{select}\NormalTok{(dplyr}\SpecialCharTok{::}\FunctionTok{contains}\NormalTok{(}\StringTok{"WFO"}\NormalTok{)) }\SpecialCharTok{\%\textgreater{}\%}\NormalTok{ dplyr}\SpecialCharTok{::}\FunctionTok{summarize\_all}\NormalTok{(}\AttributeTok{.funs =}\NormalTok{ sum)}
\end{Highlighting}
\end{Shaded}

\begin{verbatim}
## # A tibble: 1 x 3
##   WFO_authorship_ambiguity WFO_infraspecific_ambiguity WFO_infraspecific_link
##                      <int>                       <int>                  <int>
## 1                      142                         462                     37
\end{verbatim}

The bulk of these flags denotes an \verb|infraspecific_ambiguity|, which
can generally be ignored, provided that the user did not manually
truncate any trinomials to binomials for input. The \(37\)
\verb|infraspecific_link| flags are likewise typically not problematic,
as these simply highlight when the input binomial differs from the
output binomial via a trinomial link at some point in the graph. The
remaining \(142\) \verb|authorship_ambiguity| are the most problematic,
as these indicate taxa that have multiple conflicting matches. These
should be manually explored and used with caution.

\noindent Instead of using a single backbone, the user can decide to use
any subset of the backbones \verb|c('BGCI', 'WFO', 'WCVP', 'GBIF')| or
use all of them by simply calling \verb|matching()| without any
argument. While \verb|matching()| considers all backbones being equally
important, the function \verb|sequential_matching()| can be used to call
\verb|matching()| for individual backbones sequentially. For every
species, the matched backbone is provided in the column
\verb|Matched.Backbone|.

\begin{Shaded}
\begin{Highlighting}[]
\NormalTok{result }\OtherTok{\textless{}{-}}\NormalTok{ fia }\SpecialCharTok{\%\textgreater{}\%} 
  \FunctionTok{sequential\_matching}\NormalTok{(}\AttributeTok{sequential\_backbones =} \FunctionTok{c}\NormalTok{(}\StringTok{\textquotesingle{}BGCI\textquotesingle{}}\NormalTok{, }\StringTok{\textquotesingle{}WFO\textquotesingle{}}\NormalTok{, }\StringTok{\textquotesingle{}WCVP\textquotesingle{}}\NormalTok{))}
\end{Highlighting}
\end{Shaded}

\noindent Remember that \verb|matching()| and
\verb|sequential_matching()| match any species in the database and thus
can provide matches to synonyms rather than accepted species. To get
only accepted species returned use \verb|resolve_synonyms()| after the
matching function.

\hypertarget{translate-species-names-between-two-databases.}{%
\subsection{Translate species names between two
databases.}\label{translate-species-names-between-two-databases.}}

Oftentimes, researches require integrating multi-modal data from
different sources for their analyses. Here, we demonstrate the use of
the function \verb|translate_trees()|, which allows a user directly
translate names from an input database to a target database. First, we
resolve both databases individually according to the single backbone
(WFO) and compare the resolved names. Then, we use translate\_trees to
translate the input species names into the target names.

\begin{Shaded}
\begin{Highlighting}[]
\NormalTok{input }\OtherTok{\textless{}{-}}\NormalTok{ tibble}\SpecialCharTok{::}\FunctionTok{tibble}\NormalTok{(}
  \AttributeTok{Genus =} \FunctionTok{c}\NormalTok{(}\StringTok{\textquotesingle{}Aria\textquotesingle{}}\NormalTok{, }\StringTok{\textquotesingle{}Ardisia\textquotesingle{}}\NormalTok{, }\StringTok{\textquotesingle{}Malus\textquotesingle{}}\NormalTok{),}
  \AttributeTok{Species =} \FunctionTok{c}\NormalTok{(}\StringTok{\textquotesingle{}umbellata\textquotesingle{}}\NormalTok{, }\StringTok{\textquotesingle{}japonica\textquotesingle{}}\NormalTok{, }\StringTok{\textquotesingle{}sylvestris\textquotesingle{}}\NormalTok{)}
\NormalTok{)}
\NormalTok{target }\OtherTok{\textless{}{-}}\NormalTok{ tibble}\SpecialCharTok{::}\FunctionTok{tibble}\NormalTok{(}
  \AttributeTok{Genus =} \FunctionTok{c}\NormalTok{(}\StringTok{\textquotesingle{}Sorbus\textquotesingle{}}\NormalTok{, }\StringTok{\textquotesingle{}Ardisia\textquotesingle{}}\NormalTok{, }\StringTok{\textquotesingle{}Malus\textquotesingle{}}\NormalTok{),}
  \AttributeTok{Species =} \FunctionTok{c}\NormalTok{(}\StringTok{\textquotesingle{}umbellata\textquotesingle{}}\NormalTok{, }\StringTok{\textquotesingle{}montana\textquotesingle{}}\NormalTok{, }\StringTok{\textquotesingle{}orientalis\textquotesingle{}}\NormalTok{)}
\NormalTok{)}
\end{Highlighting}
\end{Shaded}

\small

\begin{Shaded}
\begin{Highlighting}[]
\NormalTok{input }\SpecialCharTok{\%\textgreater{}\%}
  \FunctionTok{matching}\NormalTok{(}\AttributeTok{backbone =} \StringTok{\textquotesingle{}WFO\textquotesingle{}}\NormalTok{) }\SpecialCharTok{\%\textgreater{}\%}
  \FunctionTok{resolve\_synonyms}\NormalTok{(}\StringTok{\textquotesingle{}WFO\textquotesingle{}}\NormalTok{) }\SpecialCharTok{\%\textgreater{}\%}
\NormalTok{  dplyr}\SpecialCharTok{::}\FunctionTok{select}\NormalTok{(}\DecValTok{1}\SpecialCharTok{:}\DecValTok{6}\NormalTok{)}
\end{Highlighting}
\end{Shaded}

\begin{verbatim}
## # A tibble: 3 x 6
##   Orig.Genus Orig.Species Matched.Genus Matched.Species Accepted.Genus
##   <chr>      <chr>        <chr>         <chr>           <chr>         
## 1 Ardisia    japonica     Ardisia       japonica        Ardisia       
## 2 Aria       umbellata    Aria          umbellata       Aria          
## 3 Malus      sylvestris   Malus         sylvestris      Malus         
## # i 1 more variable: Accepted.Species <chr>
\end{verbatim}

\begin{Shaded}
\begin{Highlighting}[]
\NormalTok{target }\SpecialCharTok{\%\textgreater{}\%}
  \FunctionTok{matching}\NormalTok{(}\AttributeTok{backbone =} \StringTok{\textquotesingle{}WFO\textquotesingle{}}\NormalTok{) }\SpecialCharTok{\%\textgreater{}\%}
  \FunctionTok{resolve\_synonyms}\NormalTok{(}\StringTok{\textquotesingle{}WFO\textquotesingle{}}\NormalTok{) }\SpecialCharTok{\%\textgreater{}\%}
\NormalTok{  dplyr}\SpecialCharTok{::}\FunctionTok{select}\NormalTok{(}\DecValTok{1}\SpecialCharTok{:}\DecValTok{6}\NormalTok{)}
\end{Highlighting}
\end{Shaded}

\begin{verbatim}
## # A tibble: 3 x 6
##   Orig.Genus Orig.Species Matched.Genus Matched.Species Accepted.Genus
##   <chr>      <chr>        <chr>         <chr>           <chr>         
## 1 Ardisia    montana      Ardisia       montana         Ardisia       
## 2 Malus      orientalis   Malus         <NA>            <NA>          
## 3 Sorbus     umbellata    Sorbus        umbellata       Sorbus        
## # i 1 more variable: Accepted.Species <chr>
\end{verbatim}

\normalsize

Resolving both sets individually leads to a mismatch - \emph{Malus
orientalis} and \emph{Malus sylvestris} were resolved to two different
names. Now let's see whether translate\_trees can be used to match all
three species: \small

\begin{Shaded}
\begin{Highlighting}[]
\FunctionTok{translate\_trees}\NormalTok{(}\AttributeTok{df =}\NormalTok{ input, }\AttributeTok{target =}\NormalTok{ target) }\SpecialCharTok{\%\textgreater{}\%} 
\NormalTok{  dplyr}\SpecialCharTok{::}\FunctionTok{select}\NormalTok{(}\DecValTok{1}\SpecialCharTok{:}\DecValTok{4}\NormalTok{) }
\end{Highlighting}
\end{Shaded}

\begin{verbatim}
## # A tibble: 3 x 4
##   Orig.Genus Orig.Species Matched.Genus Matched.Species
##   <chr>      <chr>        <chr>         <chr>          
## 1 Ardisia    japonica     Ardisia       montana        
## 2 Aria       umbellata    Sorbus        umbellata      
## 3 Malus      sylvestris   Malus         orientalis
\end{verbatim}

\normalsize

Essentially, all three species names can be translated from the input
set to the target set. Incorporating the knowledge of the desired target
names, the function leverages the information about synonym-accepted
relations in the three backbones WFO, WCVP and GBIF and is able to
translate \emph{Malus sylvestris} into \emph{Malus orientalis}.

\end{document}
